\documentclass[portuges,a4paper]{article}
\RequirePackage[a4paper,top=2cm,left=2cm,right=2cm,bottom=2.5cm]{geometry}

\usepackage{babel}
\usepackage[latin1]{inputenc}
\usepackage{t1enc}
\usepackage{url}
\usepackage{fancyvrb}
\usepackage{graphicx}
\usepackage{pdfpages}
\usepackage{latexsym}
\begin{document}

\fvset{fontsize=\small, numbers=left, frame=leftline, numberblanklines=false}

\title{Makefiles...}
\author{Jos� Jo�o Dias de Almeida}
\date{vers�o de \today}
\maketitle

%% \tableofcontents

=Importa Makefile em modo de Grafo

usando o seguinte c�digo,
\begin{Verbatim}
\begin{import_makefileg}[root=all,scale=0.5]
a : recolha c
	merge

recolha : d e
	analise 

c : d e f
	concilia 

all: a
	golpe final
\end{import_makefileg}
\end{Verbatim}

obt�m-se:

\begin{import_makefileg}[root=all,scale=0.5]
a : recolha c
	merge

recolha : d e
	analise 

c : d e f
	concilia 

all: a
	golpe final
\end{import_makefileg} 

e com trim\_mode=1

\begin{import_makefileg}[root=all,scale=0.5,trim_mode=1]
a : recolha c
	merge

recolha : d e
	analise 

c : d e f
	concilia 

all: a
	golpe final
\end{import_makefileg}

E em modo comando:
 .  com rankdir = 0:

\import_makefileg[trim_mode=1,rankdir=0]{makeex}
 .  com rankdir = 1:

\import_makefileg[trim_mode=1]{makeex}

 .  com rankdir = 0, trimmode=0:
\import_makefileg[trim_mode=0,rankdir=0]{makeex}

 .  com rankdir = 1 trimmode=0:
\import_makefileg[trim_mode=0,rankdir=1]{makeex}

#

\end{document}
